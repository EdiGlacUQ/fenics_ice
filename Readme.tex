\documentclass[11pt, reqno, nocenter]{article}


\usepackage{geometry}                % See geometry.pdf to learn the layout options. There are lots.
\geometry{letterpaper}                   % ... or a4paper or a5paper or ... 
%\geometry{landscape}                % Activate for for rotated page geometry
%\usepackage[parfill]{parskip}    % Activate to begin paragraphs with an empty line rather than an indent
\usepackage{graphicx}
\usepackage{amssymb}
\usepackage{epstopdf}
\usepackage{amsmath}
\newcommand{\R}{\mathbb{R}}
\newcommand{\Rho}{\mathrm{P}}
\usepackage{hyperref}
\usepackage{parskip}
\usepackage{verbatim}
\usepackage{dirtree}

\DeclareGraphicsRule{.tif}{png}{.png}{`convert #1 `dirname #1`/`basename #1 .tif`.png}



\title{Fenics Ice Sheet Model Readme}
%\date{}                                           % Activate to display a given date or no date

\begin{document}
\maketitle

This document briefly outlines how to get started with the Fenics ice sheet model. 


\section{Installation}

The ice sheet mdoel is built using the open source Python finite element software Fenics, and depends on the package tlm-adjoint for implementing inversion and error propagation capabilities. The simplest way to install Fenics and tlm-adjoint is to create a conda environment. 

\subsection{Installing Fenics}
 
1. Install Anaconda. This can be either Anaconda itself, or miniconda, which is a stripped down version. Ensure the Python version is greater than 3.6. Installer can be found here: https://www.anaconda.com/distribution/ 

2. Create a new conda environment. \\
conda create -n fenics -c conda-forge fenics fenics-dijitso fenics-dolfin fenics-ffc fenics-fiat fenics-libdolfin fenics-ufl

3. Enter the conda environment: \\
source activate fenics

4. Make sure the pip package manager is up to date: \\
pip install -{}-upgrade pip

5. Install the following packages: \\
conda install matplotlib numpy ipython scipy 

6. Install hdf5 for python: \\
\url{http://docs.h5py.org/en/latest/index.html} \\
pip install h5py

7. Install pyrevolve: \\
\url{https://github.com/opesci/pyrevolve}

Change to directory where you would like to download pyrevolve to. You can delete the pyrevolve directory after finishing this step. \\ \\
git clone \url{https://github.com/opesci/pyrevolve.git} \\
cd pyrevolve/ \\
python setup.py install

8. Install mpi4py: \\
\url{http://mpi4py.scipy.org/docs/} \\
pip install mpi4py

9. To enter this environment: \\
source activate fenics

10. To exit: \\
source deactivate fenics

\subsection{tlm\_adjoint}

1. Clone the git repository to the local drive where you want it to live:\\
git clone \url{https://github.com/jrmaddison/tlm_adjoint.git}

\subsection{Fenics ICE}

1. Clone the git repository to the local drive where you want it to live: \\
git clone \url{https://github.com/cpk26/fenics_ice.git}

\subsection{Modifying the Python Path}

Modify the default paths python looks for modules to include tlm\_adjoint and fenics ice. Add to the end of .bashrc: \\
PYTHONPATH=\"\$\{PYTHONPATH\}:/PATH/TO/tlm\_adjoint/python:/PATH/TO/fenics\_ice/code\" \\
export PYTHONPATH

\section{Program structure}

\subsection{Directory Structure}

\dirtree{%
.1 fenics\_ice.
.2 code.
.3 model.py.
.3 solver.py.
.2 runs.
.3 process\_eigendec.py.
.3 run\_balancemeltrates.py.                                                                                              
.3 run\_eigendec.py.                                                                                                      
.3 run\_errorprop.py.                                                                                                      
.3 run\_forward.py.                                                                                                        
.3 run\_inv.py.                                                                                                  
.3 run\_invsigma.py.                                                                                                    
.3 run\_momsolve.py.
.2 scripts.
.3 ismipc.
.2 aux.
.3 gen\_ismipC\_domain.py.
.3 test\_domains.py.
.2 input.
.3 ismipc.
.2 output.
.3 ismipc.
}

\subsection{Overview}

The core of the ice sheet model is in two files: {\tt /code/model.py} and {\tt /code/solver.py}. These are utilized by the python scripts in the {\tt /runs} folder, which execute specific parts of a simulation. The scripts there are generic to any simulation. Each specific simulation then has its own primary folder in the {\tt /scripts} folder, which will call program files in {tt /runs} with specific parameters and data files.

The bash scripts in {\tt /scripts} are where parameters and data file locations are specified. The data and parameters are used by the program files in {\tt /runs} to create a model object (via a class defined in {\tt model.py}) and subsequently a solver object (via a class defined in {\tt solver.py}). The model object contains all the necessary data for a simulation, such as topography, constants, and velocity observations for inversions. The solver object contains the ice sheet physics/inversion code. The model object is passed as a parameter to your solver object. This object then allows you to solve the SSA equations on your domain, invert for basal drag or $B_{glen}$, and perform uncertainty quantification. 


The {\tt /aux} folder contains auxillary files; in here, the file {\tt gen\_ismipC\_domain.py} generates the ismipC domain,  based off definitions in {\tt test\_domains.py}. The {\tt /input} folder is where input files, such as topography and ice thickness, for specific simulations are located. Similarily, the {\tt /output} folder is where output is stored from specific simulations.

\section{Tutorial: A walkthrough IsmipC}

\subsection{Generate synthetic domain}







\end{document}  