\documentclass[11pt, reqno, nocenter]{article}


\usepackage{geometry}                % See geometry.pdf to learn the layout options. There are lots.
\geometry{letterpaper}                   % ... or a4paper or a5paper or ... 
%\geometry{landscape}                % Activate for for rotated page geometry
%\usepackage[parfill]{parskip}    % Activate to begin paragraphs with an empty line rather than an indent
\usepackage{graphicx}
\usepackage{amssymb}
\usepackage{epstopdf}
\usepackage{amsmath}
\newcommand{\R}{\mathbb{R}}
\newcommand{\Rho}{\mathrm{P}}
\usepackage{hyperref}
\usepackage{parskip}
\usepackage{spverbatim}
\usepackage{dirtree}
\usepackage{bm}
\usepackage{siunitx}

\DeclareGraphicsRule{.tif}{png}{.png}{`convert #1 `dirname #1`/`basename #1 .tif`.png}



\title{Fenics Ice Sheet Model User Guide}
%\date{}                                           % Activate to display a given date or no date

\begin{document}

\maketitle

This document briefly outlines how to get started with the Fenics ice sheet model. 

\tableofcontents 
\section{Installation}

The ice sheet mdoel is built using the open source Python finite element software Fenics, and depends on the package tlm-adjoint for implementing inversion and error propagation capabilities. The simplest way to install Fenics and tlm-adjoint is to create a conda environment. 

\subsection{Installing Fenics}
 
1. Install Anaconda. This can be either Anaconda itself, or miniconda, which is a stripped down version. Ensure the Python version is greater than 3.6. Installer can be found here: https://www.anaconda.com/distribution/ 

2. Create a new conda environment. \\
conda create -n fenics -c conda-forge fenics fenics-dijitso fenics-dolfin fenics-ffc fenics-fiat fenics-libdolfin fenics-ufl

3. Enter the conda environment: \\
source activate fenics

4. Make sure the pip package manager is up to date: \\
pip install -{}-upgrade pip

5. Install the following packages: \\
conda install matplotlib numpy ipython scipy 

6. Install hdf5 for python: \\
\url{http://docs.h5py.org/en/latest/index.html} \\
pip install h5py

7. Install pyrevolve: \\
\url{https://github.com/opesci/pyrevolve}

Change to directory where you would like to download pyrevolve to. You can delete the pyrevolve directory after finishing this step. \\ \\
git clone \url{https://github.com/opesci/pyrevolve.git} \\
cd pyrevolve/ \\
python setup.py install

8. Install mpi4py: \\
\url{http://mpi4py.scipy.org/docs/} \\
pip install mpi4py

9. To enter this environment: \\
source activate fenics

10. To exit: \\
source deactivate fenics

\subsection{Installing tlm\_adjoint}

1. Clone the git repository to the local drive where you want it to live:\\
git clone \url{https://github.com/jrmaddison/tlm_adjoint.git}

\subsection{Installing Fenics Ice}

1. Clone the git repository to the local drive where you want it to live: \\
git clone \url{https://github.com/cpk26/fenics_ice.git}


\subsection{Creating environment variables}
Create an environment variable storing the fenics\_ice base directory by adding the following to .bashrc, amending the path appropriately for your system. \\
FENICS\_ICE\_BASE\_DIR="/XXXX/XXXX/XXXX/fenics\_ice" \\                                            
export FENICS\_ICE\_BASE\_DIR   


\subsection{Modifying the Python Path}

Modify the default paths python looks for modules to include tlm\_adjoint and fenics ice. Add to the end of .bashrc: \\
PYTHONPATH=\"\$\{PYTHONPATH\}:/PATH/TO/tlm\_adjoint/python:/PATH/TO/fenics\_ice/code\" \\
export PYTHONPATH

\section{Program structure}

\subsection{Directory Structure}

\dirtree{%
.1 fenics\_ice.
.2 code.
.3 model.py.
.3 solver.py.
.2 runs.
.3 process\_eigendec.py.
.3 run\_balancemeltrates.py.                                                                                              
.3 run\_eigendec.py.                                                                                                      
.3 run\_errorprop.py.                                                                                                      
.3 run\_forward.py.                                                                                                        
.3 run\_inv.py.                                                                                                  
.3 run\_invsigma.py.                                                                                                    
.3 run\_momsolve.py.
.2 scripts.
.3 ismipc.
.2 aux.
.3 gen\_ismipC\_domain.py.
.3 test\_domains.py.
.2 input.
.3 ismipc.
.2 output.
.3 ismipc.
}

\subsection{Overview}

The core of the ice sheet model is in two files: {\tt /code/model.py} and {\tt /code/solver.py}. These are utilized by the python scripts in the {\tt /runs} folder, which execute specific parts of a simulation. The python scripts there are generic to any simulation. Each specific simulation then has its own primary folder in the {\tt /scripts} folder, with simple bash scripts which call program files in {\tt /runs} with specific parameters and data files.

The bash scripts in {\tt /scripts} are where parameters and data file locations are specified; these are bash simple wrapper scripts for calling python scripts in {\tt /runs}. The data and parameters are used by the program files in {\tt /runs} to create a model object (via a class defined in {\tt model.py}) and subsequently a solver object (via a class defined in {\tt solver.py}). The model object contains all the necessary data for a simulation, such as topography, constants, and velocity observations for inversions. The solver object contains the ice sheet physics/inversion code. The model object is passed as a parameter to your solver object. This object then allows you to solve the SSA equations on your domain, invert for basal drag or $B_{glen}$, and perform uncertainty quantification. The options of any python script in the {\tt /runs} folder can be viewed by typing 'python run\_xxx.py --help'.


The {\tt /aux} folder contains auxillary files; in here, the file {\tt gen\_ismipC\_domain.py} generates the ismipC domain,  based off definitions in {\tt test\_domains.py}. The {\tt /input} folder is where input files, such as topography and ice thickness, for specific simulations are located. Similarily, the {\tt /output} folder is where output is stored from specific simulations.

\section{Tutorial: A Walkthrough of IsmipC}

The Ice Sheet Model Intercomparison Project for Higher-Order ice sheet Models (ISMIP) provides a standardized set of idealized tests for ice sheet models. In this walkthrough, we apply Fenics\_ice to the domain prescribed by experiment C (IsmipC). The original IsmipC is a static simulation, meaning time evolution is not considered. We'll extend it by running a dynamic simulation for the purposes of performing uncertainty quantification.

\section{IsmipC}

\subsection{Generating the Domain}

Navigate to the {\tt /fenics\_ice} base directory. Activate the fenics conda environment.

\begin{verbatim}
> source activate fenics 
\end{verbatim}

To begin, we'll generate the synthetic domain defined by the IsmipC experiment. The specifications are coded in the file {\tt /aux/test\_domains.py}. We'll use the python script 'gen\_ismipC\_domain.py' to create a domain with a given length and resolution.

\begin{verbatim}
> cd $FENICS_ICE_BASE_DIR/aux 
> python gen_ismipC_domain.py -o ../input/ismipC -L 40000 -nx 100 -ny 100 
\end{verbatim}

This will generate a square domain with side-length 40\si{\kilo\metre}, at resolution of 100 x 100, placing the output in the folder {\tt input/ismipc}.
Let's observe the files that are generated.

\begin{spverbatim}
> ls $FENICS_ICE_BASE_DIR/input/ismipC
B2.xml  Bglen.xml  alpha.xml  bed.xml  bmelt.xml  grid_data.npz  mask.xml  mesh.xml  smb.xml  thick.xml 
\end{spverbatim}

The .xml files contain discretized scalar fields over the IsmipC domain on a FEniCS mesh. The extension .npz indicates a numpy file format, and contains the domain resolution and length.

\begin{itemize}
\item B2.xml -- $\beta^2$ coefficient for linear sliding law $(\bm{\tau_b} = \beta^2 \bm{u})$;
\item  Bglen.xml -- parameter in Glen's flow law
\item alpha.xml -- variable in sliding law 
\item bed.xml -- basal topography
\item bmelt.xml -- basal melt. 
\item mask.xml -- mask of our domain
\item mesh.xml -- FEniCS mesh
\item smb.xml -- surface mass balance
\item thick.xml -- ice thickness

\end{itemize}

\subsection{Solving the Momentum Equations}

Having generated the files which describe our domain, we can solve the SSA momentum equations to determine ice velocities. 

\begin{spverbatim}
> cd $FENICS_ICE_BASE_DIR/scripts/ismipc/ 
> ./forward_solve.sh

Generating new mesh                                                                                                                                              
Building point search tree to accelerate distance queries.                                                                                                       
Computed bounding box tree with 39999 nodes for 20000 points.                                                                                                    
Solving nonlinear variational problem.                                                                                                                             
Newton iteration 0: r (abs) = 1.585e+03 (tol = 1.000e-08) r (rel) = 1.000e+00 (tol = 5.000e-02)                                                                 
Newton iteration 1: r (abs) = 1.139e+02 (tol = 1.000e-08) r (rel) = 7.186e-02 (tol = 5.000e-02)                                                                 
Newton iteration 2: r (abs) = 1.307e+02 (tol = 1.000e-08) r (rel) = 8.248e-02 (tol = 5.000e-02)                                                                  
Newton iteration 3: r (abs) = 9.443e+01 (tol = 1.000e-08) r (rel) = 5.958e-02 (tol = 5.000e-02)                                                                  
Newton iteration 4: r (abs) = 5.682e+01 (tol = 1.000e-08) r (rel) = 3.585e-02 (tol = 5.000e-02)                                                                 
Newton solver finished in 5 iterations and 5 linear solver iterations.                                                                                         
Solving nonlinear variational problem.                                                                                                                          
Newton iteration 0: r (abs) = 6.650e+01 (tol = 1.000e-05) r (rel) = 1.000e+00 (tol = 1.000e-05)                                                                 
Newton iteration 1: r (abs) = 4.913e+00 (tol = 1.000e-05) r (rel) = 7.387e-02 (tol = 1.000e-05)                                                                  
Newton iteration 2: r (abs) = 4.393e-02 (tol = 1.000e-05) r (rel) = 6.606e-04 (tol = 1.000e-05)                                                                 
Newton iteration 3: r (abs) = 5.647e-06 (tol = 1.000e-05) r (rel) = 8.492e-08 (tol = 1.000e-05)                                                                 
Newton solver finished in 4 iterations and 4 linear solver iterations.                                                                                           
Time for solve: 4.667648553848267      

ls $FENICS_ICE_BASE_DIR/input/ismipC/momsolve
...
\end{spverbatim}

The script automatically places the output in the subdirectory of {\tt input/} since we'll use the velocities we solved for in the next step, generating synthetic observations.

Opening {\tt forward\_solve.sh} with any text editor, we can confirm that this is a simple wrapper script.

\begin{spverbatim}
#!/bin/bash
set -e

BASE_DIR=$FENICS_ICE_BASE_DIR
RUN_DIR=$BASE_DIR/runs

INPUT_DIR=$BASE_DIR/input/ismipC
OUTPUT_DIR=$INPUT_DIR/momsolve

cd $RUN_DIR

python run_momsolve.py -b -q 0 -d $INPUT_DIR -o $OUTPUT_DIR
\end{spverbatim}

The bash script specifies key folders, that we are solving momentum equations on a domain with periodic boundary conditions (-b option), and that we are using a linear sliding law (-q 0).

\subsection{Generating Synthetic Observations}

IsmipC is a synthetic experiment, meaning we don't have observational data of ice velocities. We can generate pseudo-oberservations by adding gaussian noise to the solved velocities. This script assumes the noise is additive rather than a multiplicative factor. The python script 'Uobs\_from\_momsolve.py' takes the vector field in U.xml and generates the files: data\_mesh.xml, data\_mask.xml, u\_obs.xml, v\_obs.xml, u\_std.xml, and v\_std.xml. The first file contains the data mesh, the second identifies where velocity data is available, the next two files contain the pseudo-observations in the x and y directions, with the final two files containing the standard deviation of the gaussian noise applied. A data\_mask.xml and data\_mask.pvd file are generated by forward\_solve.sh since these are duplicates of the domain mask. 

\begin{spverbatim}
> cd $FENICS_ICE_BASE_DIR/aux/
> python Uobs_from_momsolve.py -b -L 40000 -d $FENICS_ICE_BASE_DIR/input/ismipC/momsolve
> find $FENICS_ICE_BASE_DIR/input/ismipC/momsolve -type f -regex '.*\(obs\|std\).xml'
/mnt/c/Users/ckozi/Documents/Python/fenics/fenics_ice/input/ismipC/u_obs.xml
/mnt/c/Users/ckozi/Documents/Python/fenics/fenics_ice/input/ismipC/u_std.xml
/mnt/c/Users/ckozi/Documents/Python/fenics/fenics_ice/input/ismipC/v_obs.xml
/mnt/c/Users/ckozi/Documents/Python/fenics/fenics_ice/input/ismipC/v_std.xml
\end{spverbatim}

Copy the five files generated into {\tt \$FENICS\_ICE\_BASE\_DIR/input/ismipC/}. A study site in Antarctica or Greenland would require generating these files and the data\_mask from a surface velocity dataset such as NSIDC MEaSUREs.

\begin{verbatim}
> cd $FENICS_ICE_BASE_DIR/input/ismipC/momsolve
> cp data_mask.xml data_mesh.xml mask_vel.xml u_*.xml v_*.xml ..
\end{verbatim}




\end{document}  